%% This is file `elsarticle-template-2-harv.tex',
%%
%% Copyright 2009 Elsevier Ltd
%%
%% This file is part of the 'Elsarticle Bundle'.
%% ---------------------------------------------
%%
%% It may be distributed under the conditions of the LaTeX Project Public
%% License, either version 1.2 of this license or (at your option) any
%% later version.  The latest version of this license is in
%%    http://www.latex-project.org/lppl.txt
%% and version 1.2 or later is part of all distributions of LaTeX
%% version 1999/12/01 or later.
%%
%% The list of all files belonging to the 'Elsarticle Bundle' is
%% given in the file `manifest.txt'.
%%
%% Template article for Elsevier's document class `elsarticle'
%% with harvard style bibliographic references
%%
%% $Id: elsarticle-template-2-harv.tex 155 2009-10-08 05:35:05Z rishi $
%% $URL: http://lenova.river-valley.com/svn/elsbst/trunk/elsarticle-template-2-harv.tex $
%%

%%\documentclass[preprint,authoryear,12pt]{elsarticle}

%% Use the option review to obtain double line spacing
%% \documentclass[authoryear,preprint,review,12pt]{elsarticle}

%% Use the options 1p,twocolumn; 3p; 3p,twocolumn; 5p; or 5p,twocolumn
%% for a journal layout:

%% Astronomy & Computing uses 5p
%% \documentclass[final,authoryear,5p,times]{elsarticle}
\documentclass[final,authoryear,5p,times,twocolumn]{elsarticle}

%% if you use PostScript figures in your article
%% use the graphics package for simple commands
%% \usepackage{graphics}
%% or use the graphicx package for more complicated commands
\usepackage{graphicx}
%% or use the epsfig package if you prefer to use the old commands
%% \usepackage{epsfig}

%% The amssymb package provides various useful mathematical symbols
\usepackage{amssymb}
%% The amsthm package provides extended theorem environments
%% \usepackage{amsthm}

\usepackage[pdftex,pdfpagemode={UseOutlines},bookmarks,bookmarksopen,colorlinks,linkcolor={blue},citecolor={green},urlcolor={red}]{hyperref}
\usepackage{hypernat}

%% The lineno packages adds line numbers. Start line numbering with
%% \begin{linenumbers}, end it with \end{linenumbers}. Or switch it on
%% for the whole article with \linenumbers after \end{frontmatter}.
%% \usepackage{lineno}

%% natbib.sty is loaded by default. However, natbib options can be
%% provided with \biboptions{...} command. Following options are
%% valid:

%%   round  -  round parentheses are used (default)
%%   square -  square brackets are used   [option]
%%   curly  -  curly braces are used      {option}
%%   angle  -  angle brackets are used    <option>
%%   semicolon  -  multiple citations separated by semi-colon (default)
%%   colon  - same as semicolon, an earlier confusion
%%   comma  -  separated by comma
%%   authoryear - selects author-year citations (default)
%%   numbers-  selects numerical citations
%%   super  -  numerical citations as superscripts
%%   sort   -  sorts multiple citations according to order in ref. list
%%   sort&compress   -  like sort, but also compresses numerical citations
%%   compress - compresses without sorting
%%   longnamesfirst  -  makes first citation full author list
%%
%% \biboptions{longnamesfirst,comma}

% \biboptions{}

\journal{Astronomy \& Computing}

%% Make single quotes look right in verbatim mode
\usepackage{upquote}

\begin{document}

\begin{frontmatter}

%% Title, authors and addresses

%% use the tnoteref command within \title for footnotes;
%% use the tnotetext command for the associated footnote;
%% use the fnref command within \author or \address for footnotes;
%% use the fntext command for the associated footnote;
%% use the corref command within \author for corresponding author footnotes;
%% use the cortext command for the associated footnote;
%% use the ead command for the email address,
%% and the form \ead[url] for the home page:
%%
%% \title{Title\tnoteref{label1}}
%% \tnotetext[label1]{}
%% \author{Name\corref{cor1}\fnref{label2}}
%% \ead{email address}
%% \ead[url]{home page}
%% \fntext[label2]{}
%% \cortext[cor1]{}
%% \address{Address\fnref{label3}}
%% \fntext[label3]{}

\title{The Future of Astronomical Data Formats II. Learning from 25
  years of the $N$-Dimensional Data Format}

%% use optional labels to link authors explicitly to addresses:
%% \author[label1,label2]{<author name>}
%% \address[label1]{<address>}
%% \address[label2]{<address>}

\author[cornell]{Tim Jenness\corref{cor1}}
\ead{tjenness@cornell.edu}
\author[jac]{David S. Berry}
\author[jac]{Malcolm J.\ Currie}
\author[noao]{Frossie Economou}
\author[aao]{Keith Shortridge}
\author[bristol]{Mark B.\ Taylor}
\author[ral]{Patrick T.\ Wallace}

\cortext[cor1]{Corresponding author}

\address[cornell]{Department of Astronomy, Cornell University, Ithaca,
  NY 14853, USA}
\address[jac]{Joint Astronomy Centre, 660 N.\ A`oh\=ok\=u Place, Hilo, HI
  96720, USA}
\address[noao]{National Optical Astronomy
  Observatory, 950 N Cherry Ave, Tucson, AZ 85719, USA}
\address[aao]{Australian Astronomical Observatory, 105 Delhi Rd, North
Ryde, NSW 2113, Australia}
\address[bristol]{H.\ H.\ Wills Physics Laboratory, Bristol University, Tyndall Avenue, Bristol, UK}
\address[ral]{Space Science and Technology Department, STFC Rutherford Appleton Laboratory, Harwell Oxford, Didcot, Oxfordshire, OX11 0QX, UK}

\begin{abstract}
%% Text of abstract

The extensible $N$-dimensional Data Format (NDF) was designed and
developed in the late 1980s to provide a data model suitable for use
in a variety of astronomy data processing applications supported by
Starlink. This paper provides an overview of the historical drivers
for the development of NDF and the lessons learned from using the
format for many years in the Starlink software collection and in data
acquisition systems.

\end{abstract}

\begin{keyword}
%% keywords here, in the form: keyword \sep keyword

%% MSC codes here, in the form: \MSC code \sep code
%% or \MSC[2008] code \sep code (2000 is the default)

data formats \sep
Starlink

\end{keyword}

\end{frontmatter}

% \linenumbers

%% Journal abbreviations
\newcommand{\mnras}{Mon Not R Astron Soc}
\newcommand{\aap}{Astron Astrophys}
\newcommand{\aaps}{Astron Astrophys Supp}
\newcommand{\pasp}{Pub Astron Soc Pacific}
\newcommand{\apj}{Astrophys J}
\newcommand{\apjs}{Astrophys J Supp}
\newcommand{\qjras}{Quart J R Astron Soc}
\newcommand{\an}{Astron.\ Nach.}
\newcommand{\ijimw}{Int.\ J.\ Infrared \& Millimeter Waves}
\newcommand{\procspie}{Proc.\ SPIE}
\newcommand{\aspconf}{ASP Conf. Ser.}

%% Applications


%% main text
\section{Introduction}
\label{sec:intro}

Author list not finalised. Should invite PTW, PWD, NXG and RFWS and others.

There is a renewed interest in file format choices for astronomy
including discussions on the future of FITS (Thomas et al 2014) and
with some projects adopting HDF5 and investigating JPEG2000. The
Starlink project \citep{2000ASSL..250...93W,2002A&G....43a..25P},
which began in 1980 \citep{1982MmSAI..53...55T}, was interested in
using a single file format for all their data reduction software. The
shift from a FITS-based format to a hierarchical self-describing
format was driven by data reduction efficiency requirements and the
desire to reuse software. A file format without a corresponding data
model caused chaos and it took sometime until order could be restored
with the definition of the $N$-Dimensional Data Format (NDF).

When the Starlink Project was created it soon became apparent that a
unified file format should be adopted for all the Starlink
applications. FITS \citep{1979ipia.coll..445W,1981A&AS...44..363W} had
recently been developed and was adopted as a tape interchange
format. A file format was required that was explicitly optimized for
data reduction applications and the choice of format involved much
debate (reference to FITS adoption). The initial Starlink software
environment, known as the INTERIM environment, proposed the use of the
Bulk Data Frame \citep[BDF;][]{1980SPIE..264...70P,SUN4}. The BDF format
was heavily influenced by FITS and used a FITS header block with
standard FITS keywords. Whilst it had a flexible design a special
``IMAGE'' variant was standardised for astronomical images and
spectra.

A data reduction file format should be efficient when generating
intermediate files and be able to address and modify individual items.
Additionally, it is import to be able to copy individual
self-contained structures from one file to another. At the time FITS
was not capable of doing this and it was realised that the finalised
Starlink software environment would benefit from a more advanced file
format capable of addressing hierarchical structures
\citep{1981STARENT4}.


\section{Hierarchical Data System}

The Starlink Data System\footnote{from which the file extension of
  \texttt{.sdf}, for Starlink Data File, was chosen.} was first
proposed in 1981 and the first version was released in 1982 \citep[see
e.g.][]{1982QJRAS..23..485D,1991STARB...8....2L}. In 1983 the name was
changed to the Hierarchical Data System (HDS) to make the file format
benefits more explicit. It was originally written in the BLISS
programming language on a VAX/VMS system and later rewritten in C and
ported to Unix.  HDS was in common usage in 1986 although it was
hampered by the next generation Starlink Software Environment
\citep{1986BICDS..30...13L}.  There are many parallels between the
hierarchical way HDS stores data within files and the way that a
filing system organises the files themselves. An early description of
the internal layout of an HDS file can be found in \citet{SSN27a}.

Some key features of the HDS design are:

\begin{itemize}
\item Hierarchical organization of arbitrary structures, including the
  ability to store arrays of structures.
\item The hierarchy is self-describing and can be queried.
\item The ability to associates structures with an arbitrary data type.
\item The ability to delete, copy or rename structures within a file.
\item Automatic byte swapping whilst retaining the native byte order
  for output files.
\end{itemize}

More recently HDS has been extended to support 64-bit file offsets so
that larger files can be written\footnote{Support for individual data
  arrays with a 64-bit integer size is not yet possible.}, and also the
addition of a 64-bit integer data type \citep{P82_adassxxiii}.

\begin{table}
\caption{HDS basic data types. The unsigned types did not correspond
  to standard Fortran~77 data types but had to be included for
  compatibility with astronomy instrumentation.}
\label{tab:hdstypes}
\begin{center}
\begin{tabular}{lll}
\hline
\_BYTE & b & Signed 8-bit integer \\
\_UBYTE & ub & Unsigned 8-bit integer \\
\_WORD & w & Signed 16-bit integer \\
\_UWORD & uw & Unsigned 16-bit integer \\
\_INTEGER & i & Signed 32-bit integer \\
\_INT64 & k &Signed 64-bit integer \\
\_LOGICAL & l & Boolean \\
\_REAL & r & 32-bit IEEE float \\
\_DOUBLE & d & 64-bit IEEE float \\
\_CHAR[$*$n] & c & 8-bit Character string \\
\hline
\end{tabular}
\end{center}
\end{table}

The advantage of HDS is that it allows many different kinds of data to
be stored in a consistent and logical fashion. It is also very
flexible, in that objects can be added or deleted whilst retaining the
logical structure. HDS also provides portability of data, so that the
same data objects may be accessed from different types of computer
despite the fact that each may actually format its files and data in
different ways.

\subsection{\label{sect:objects}HDS Objects}

HDS files are known as \emph{container files} and by default have the
extension `.sdf'.  HDS files contain \emph{data objects} which will
often be referred to simply as \emph{objects}. An object is an entity
which contains data or other objects. This is the basis of the
hierarchical nature of HDS and is analogous to the concepts of
\emph{file} and \emph{directory} -- a directory can contain files and
directories which can themselves contain files and directories and so
on.

An HDS object possesses a number of attributes, each of which is
described in more detail below:

\subsubsection{\label{sect:name}Name}

The primary way of identifying an HDS object is by its \emph{name},
which must be unique within its own container object.  The name of an
object is a character string which may contain any printing
characters; white space is ignored and alphabetic characters are
capitalised. The maximum length of an HDS name is 15 characters.

There are no special rules governing the first character
({i.e.\ it can be numeric), so HDS itself allows great freedom in specifying
names (and also types -- see \S\ref{sect:type}). In practice,
however, some restrictions will be imposed by considerations of
portability of data and applications, and of possible syntax conflicts
with the environment within which HDS is used.

\subsubsection{\label{sect:type}Type}

The \emph{type} of an HDS object falls into two \emph{classes} named
Structure and Primitive.

Structure objects contain other objects called
\emph{components}. Primitive objects contain only numeric, character,
or logical values. Objects in the different classes are referred to as
\emph{structures} and \emph{primitives} while the more general term
\emph{object} refers to either a \emph{structure} or a
\emph{primitive}. Structures are analogous to the directories in a
filing system -- they can contain a part of the hierarchy below
them. Primitives are analogous to files -- they are at the bottom of
any branch of the structure and contain the actual data.

In HDS, structure types are represented by character strings with the
same rules of formation as \emph{name} (\S\ref{sect:name}) except that
a structure type may not start with an underscore character `\_' (a
structure type may also be completely blank). Examples of structure
types are `IMAGE', `SPECTRUM', `INSTR\_RESP', \emph{etc.} These do not
begin with an underscore, so they are easily distinguished from the
primitive types, which do.

Special rules apply to the primitive types, which all begin with an
underscore and are ``pre-defined'' by HDS, as shown in table
\ref{tab:hdstypes}.

\subsubsection{\label{sect:shape}Shape}

Every HDS object has a \emph{shape} or dimensionality. This is
described by an integer (the number of dimensions) and an integer
array (containing the size of each dimension). A \emph{scalar} (for
example a single number) has, by convention, a dimensionality of zero;
\emph{i.e.}\ its number of dimensions is 0. A \emph{vector} has a
dimensionality of 1; \emph{i.e.}\ its number of dimensions is 1, and
the first element of the dimension array contains the size of the
vector.  An \emph{array} refers to an object with 2 or more dimensions;
a maximum of 7 dimensions are allowed. Objects may be referred to as
\emph{scalar primitives} or \emph{vector structures} and so on.

\subsubsection{State}

The \emph{state} of an HDS object specifies whether or not its value is
defined. It is represented as a logical value where .TRUE.\ means
defined and .FALSE.\ means undefined.

Objects start out undefined when they are created and become defined
when you write a value to them. In general, an error will result if
you attempt to obtain the value of an object while it is still
undefined.


\section{File Format without Data Model}

HDS files allowed people to arrange their data in the files however
they pleased and placed no constraints on the organization of the
structures or the semantic meaning of the content. This resulted in
serious interoperability issues when moving files between applications
that nominally could read HDS files. Within the Starlink ecosystem
there were at least three prominent attempts at providing a data model
to organize the chaos.

\subsection{Wright-Giddings IMAGE}

\begin{figure*}
\begin{minipage}{\textwidth}
\begin{quote}
\small
\begin{verbatim}
HORSEHEAD  <IMAGE>

   DATA_ARRAY(384,512)  <_REAL>   100.5,102.6,110.1,109.8,105.3,107.6,
                                  ... 123.1,117.3,119,120.5,127.3,108.4
   TITLE          <_CHAR*72>      'KAPPA - Flip'
   DATA_MIN       <_REAL>         28.513
   DATA_MAX       <_REAL>         255.94
   AXIS1_DATA     <_REAL}         1,2,3,4,5,6,7,8,
                                  ... 383, 384
   AXIS1_LABEL    <_CHAR*5>       'XAXIS'
\end{verbatim}
\end{quote}
\caption{Example IMAGE structure using the Wright-Giddings
  convention. The components are all at a single level without any hierarchy.}
\end{minipage}
\end{figure*}

An early proposal \citep[][but see also \citet{SGP38}]{WrightGiddings1983} introduced the
\texttt{IMAGE} organizational scheme. This Wright-Giddings design specified that
data should go into a \texttt{DATA\_ARRAY} item and there should also be
items for pre-computed data minimum and maximum, and also a value for
a array-specific blank value. Errors were represented as standard
deviations and stored in \texttt{DATA\_ERROR} and bad-pixel masks were
stored in \texttt{DATA\_QUALITY}. Software was provided to convert BDF
format files to HDS using this model \citep{SUN96} and the format became
reasonably popular because of its simplicity.

There were a number of shortcomings with the design, not the least of
which was that it did not make use of hierarchical structures. The
design was very flat and heavily influenced by FITS and BDF.

\subsection{Figaro DST}

The Figaro data reduction package
\citep{1988igbo.conf..448C,1993ASPC...52..219S} independently adopted
a hierarchical design based on HDS, influenced by early discussions on
NDF. This DST\footnote{The reason for the name has been lost in the
  mists of time but our best guess is that it stood for \emph{\textbf{D}ata
  \textbf{ST}ructures}.} format makes good use of structures and supported
standard deviations for errors. The main image/spectral data was
labeled \texttt{Z} and was of type \texttt{IMAGE}, and axis
information was stored in structures labeled \texttt{X} and
\texttt{Y}. The axis structures reused components from the
\texttt{IMAGE} structure although they were themselves given a type of
\texttt{AXIS}. FITS-style keyword/value pairs were encoded explicitly
in a structure of type \texttt{FITS} but using scalar components for
each header item. This design meant that FITS comments could not
easily be encoded. An example trace can be found in \ref{fig:dst}.

{\color{red} How did Figaro DST files store 3d data?}

\begin{figure*}
\begin{minipage}{\textwidth}
\begin{quote}
\small
\begin{verbatim}
OUT  <FIGARO>
   Z              <IMAGE>         {structure}
      DATA(310,19)   <_REAL>         1655.552,1376.111,1385.559,1746.966,
                                     ... 1513.654,1465.343,1446.902,*,*,*,*,*
      UNITS          <_CHAR*32>      'A/D numbers per exposure'
      LABEL          <_CHAR*32>      'OBJECT - DARK'
      ERRORS(310,19)  <_REAL>        9.330093,4.624712,1.043125,3.801913,
                                     ... 16.92331,11.49692,10.9114,*,*,*,*,*

   OBS            <OBS>           {structure}
      OBJECT         <_CHAR*32>      'krypton singlet'

   X              <AXIS>          {structure}
      DATA(310)      <_REAL>         1.85115,1.852412,1.853675,1.854938,
                                     ... 2.237606,2.238869,2.240132,2.241395
      LABEL          <_CHAR*32>      'Estimated wavelength'
      UNITS          <_CHAR*32>      'microns'

   Y              <AXIS>          {structure}
      DATA(19)       <_REAL>         0.5,1.5,2.5,3.5,4.5,5.5,6.5,7.5,8.5,
                                     ... 12.5,13.5,14.5,15.5,16.5,17.5,18.5

   FITS           <FITS>          {structure}
      INSTRUME       <_CHAR*8>       'CGS4'
      TELESCOP       <_CHAR*8>       'UKIRT'
      SOFTWARE       <_CHAR*8>       'CF v1.0'
\end{verbatim}
\end{quote}
\caption{Partial dump of the structure of a DST file. Structures make
  use of a hierarchy and reuse concepts in the data array and axis
  definition. This example is from a CGS4 observation from January
  1991.}
\label{fig:dst}
\end{minipage}
\end{figure*}

\subsection{Asterix}

{\color{red} My examples are from the early 1990s so I'm not sure how
  much of this influenced NDF or was influenced by NDF. Documents from
  the mid 1980s suggest that ASTERIX used an HDS file before NDF and
  were the first to adopt HISTORY.}

The \textsc{Asterix} X-Ray data reduction package \citep{SUN98,1992STARB...9....3S} used the HDS
format exclusively until the introduction of an abstract data access
interface \citep{1995ASPC...77..199A} which allowed for the use of HDS
and FITS format files. \textsc{Asterix} defined its
own data models and layered those on top of HDS. The \texttt{SPECTRA}
layout was closest to that found in NDF and an example is shown in
Fig.\ \ref{fig:asterix} with a trace from a file created in 1992. The
use of \texttt{HISTORY}, \texttt{AXIS} and \texttt{DATA\_ARRAY}
structures provides some similarity to NDF. {\color{red} What
  components did ASTERIX develop themselves?}

\begin{figure*}
\begin{minipage}{\textwidth}
\begin{quote}
\small
\begin{verbatim}
RZ_PSPC_Z00000  <SPECTRA>

   AXIS(3)        <AXIS>          {array of structures}

   Contents of AXIS(1)
      DATA_ARRAY(729)  <_REAL>       0.07115,0.07154,0.07162001,0.07166,
                                     ... 2.97,2.975,2.98,2.985,2.99,2.995,3

   EMBOL_UNITS    <_CHAR*16>      'keV/cm**3/s'
   HISTORY        <HISTORY>       {structure}
      CREATED        <_CHAR*18>      '16-JUN-92 01:07:08'
      RECORDS(10)    <HIST_REC>      {array of structures}

      Contents of RECORDS(1)
         DATE           <_CHAR*30>      '   16-JUN-92 01:07:08'
         COMMAND        <_CHAR*35>      'RS_CREATE V1.3        [1 Jun 92]'

      EXTEND_SIZE    <_INTEGER>      10
      CURRENT_RECORD  <_INTEGER>     2

   DATA_ARRAY(729,2,162)  <_REAL>   10183284,2089223,1044320,1044320,
                                    ... 18350.33,18349.45,18275.11,36474.31
   TMIN           <_REAL>         0.017558
   TMAX           <_REAL>         29.13903
   TLSTEP         <_REAL>         0.02
   ENERGY_BOUNDS(730)  <_REAL>    0.07115,0.07154,0.07162001,0.07166,
                                  ... 2.97,2.975,2.98,2.985,2.99,2.995,3,3.01
   REDSHIFT       <_REAL>         0
   EMBOL(2,162)   <_REAL>         1.2466619E9,3.2856078E10,1.1444687E9,
                                  ... 2.8344144E8,2.3983852E9,2.8333168E8
\end{verbatim}
\end{quote}
\caption{Partial dump of the structure of an ASTERIX HDS
  \texttt{spectra} file. \texttt{HISTORY} and \texttt{AXIS} structures
  make extensive use of hierarchy and arrays of structures.  This
  example comes from the ASTERIX source distribution from 1992.}
\label{fig:asterix}
\end{minipage}
\end{figure*}

{\color{red} Were there any other structures used with HDS other than DST and IMAGE prior to NDF?}

\section{\emph{N}-Dimensional Data Format}

There is text in SGP/38 that can probably be brought over verbatim.

Funnily enough this seems to be a data model and not a data format.

Interestingly \citet{1993ASPC...52..199B} talks of how useful NDF was in designing the
2dF data reduction system using object-oriented fortran with NDF as
the object backing store.

Not trying to be comprehensive covering all possible astronomy
data. Trying to be generically useful whilst providing an extension facility.

Earliest reference to NDF I have found is Starlink bulletin 1988,
number 2. \citep{1988STARB...2...11C}

Anatomy of an NDF file

Proper block diagram of the hierarchy for each structure type? That
was part of the ADASS poster that could not be squeezed into the ADASS
paper and it needs to be made explicit.

\subsection{Data Arrays}

Data, Variance, Quality

Quality labels.

\subsubsection{Data compression}

Short description of options \citep{2008ASPC..394..650C}

\subsection{Character Attributes}

Title, Label and units

\subsection{Axes and WCS}

First there was \texttt{AXIS} and then there was AST. \citep{1998ASPC..145...41W,2001ASPC..238..129B}

\subsection{History}

Structured with a supporting API.

\subsection{Extensions}

\texttt{MORE}

FITS headers as single character block.

\subsubsection{Provenance}

Technically an extension rather than part of NDF. Or talk in context
of history?

Describe how we store the provenance and how this evolved from ascii
format for performance reasons. We also have scaling difficulties.

\section{Data Acquisition}

Figaro had a strong influence on the infrared spectroscopy community
in the United Kingdom and the United Kingdom Infrared Telescope
(UKIRT) adopted the DST format for CGS3 and CGS4. NDF was adopted in
1995 although each instrument had a slightly different flavor of
metadata and handling of multiple exposures involved distinct data
files. A unified UKIRT NDF format for all instruments, involving HDS
containers of NDF structures to handle multiple exposures, was
adopted with the release of the ORAC system.

The James Clerk Maxwell Telescope (JCMT) initially used a proposed
submillimeter standard format known as the Global Section Datafile
\citep[GSD;][formerly General Single Dish Data]{sun229}. In 1996 SCUBA
was delivered using the NDF format and a unified NDF raw data format
was adopted for ACSIS and SCUBA-2 from 2006.

{\color{red}AAO? ING?}

\section{Lessons Learned}

Covered somewhat in ADASS paper so need some more content as this is
the justification for the existence of this paper.

Successes:

\begin{itemize}
\item demonstrated usability in many wavelength regimes
\item can round-trip to and from FITS \citep[see e.g.][]{1997STARB..19...14C}
\item Pragmatic adoption of FITS header structure did make the format
  more transparent with the downside being that libraries had to be
  used to parse the character array and keep up with changes to the
  FITS convention.
\item Addition of data compression, WCS objects and provenance has
  proven the flexibility of the early design.
\end{itemize}

Things to be improved upon:

\begin{itemize}
\item Quality array should support more than a byte mask
\item Tables are important
\item Flexible variance definitions. Covariance: See Meyerdierks 1991,
  Starlink Bulletin 8, 19
\item Be vigilant with too much complexity in extensions when reusing
  an NDF structure in an extension can simplify tool access.
\item Checksumming. The FITS DATASUM facility is very useful and NDF
  should support it. Ideally it should be possible to generate a
  reference checksum for a structure. It may be that this has to be
  done in conjunction with HDS.
\item Character encodings.
\end{itemize}


\section{Conclusions}

From its beginnings in the mid-1980s the NDF data model has been used
throughout the Starlink software collection within diverse
applications such as SMURF \citep{2013MNRAS.430.2545C}, CCDPACK, GAIA
and KAPPA. It has also been used as a raw data acquisition format at the
James Clerk Maxwell Telescope and the United Kingdom Infrared
Telescope (was it used at ING and AAO?) and available in that form
from the UKIRT and JCMT archives at the Canadian Astronomy Data Centre
\citep{2008ASPC..394..450E,P01_adassxxiii}. The shift from arbitrary
use of hierarchical structures to a data model enforced by a library
and API is extremely important.

Mention HDX \citep{2003ASPC..295..221G}.

Consider reimplementing NDF on top of HDF5. \texttt{MORE} extension
requires that initially this is done by reimplementing HDS itself as a
layer on top of HDF5. Then all the Starlink software would still work
and even tools like \textsc{hdstrace}. Would allow for an easy way to
add table support to NDF.

The main impediment to further NDF take up is that it's in
Fortran. Rewriting NDF and ARY in C would take some effort.

\section{Acknowledgments}

This research has made use of NASA's Astrophysics Data System.
The Starlink software is currently maintained by the Joint Astronomy
Centre, Hawaii with support from the UK Science and Technology
Facilities Council.

The source code for the NDF library and the Starlink software
(ascl:1110.012) is open-source and is available on github at
\htmladdnormallink{https://github.com/Starlink}.

%% The Appendices part is started with the command \appendix;
%% appendix sections are then done as normal sections
%% \appendix

%% \section{}
%% \label{}

%% References
%%
%% Following citation commands can be used in the body text:
%%
%%  \citet{key}  ==>>  Jones et al. (1990)
%%  \citep{key}  ==>>  (Jones et al., 1990)
%%
%% Multiple citations as normal:
%% \citep{key1,key2}         ==>> (Jones et al., 1990; Smith, 1989)
%%                            or  (Jones et al., 1990, 1991)
%%                            or  (Jones et al., 1990a,b)
%% \cite{key} is the equivalent of \citet{key} in author-year mode
%%
%% Full author lists may be forced with \citet* or \citep*, e.g.
%%   \citep*{key}            ==>> (Jones, Baker, and Williams, 1990)
%%
%% Optional notes as:
%%   \citep[chap. 2]{key}    ==>> (Jones et al., 1990, chap. 2)
%%   \citep[e.g.,][]{key}    ==>> (e.g., Jones et al., 1990)
%%   \citep[see][pg. 34]{key}==>> (see Jones et al., 1990, pg. 34)
%%  (Note: in standard LaTeX, only one note is allowed, after the ref.
%%   Here, one note is like the standard, two make pre- and post-notes.)
%%
%%   \citealt{key}          ==>> Jones et al. 1990
%%   \citealt*{key}         ==>> Jones, Baker, and Williams 1990
%%   \citealp{key}          ==>> Jones et al., 1990
%%   \citealp*{key}         ==>> Jones, Baker, and Williams, 1990
%%
%% Additional citation possibilities
%%   \citeauthor{key}       ==>> Jones et al.
%%   \citeauthor*{key}      ==>> Jones, Baker, and Williams
%%   \citeyear{key}         ==>> 1990
%%   \citeyearpar{key}      ==>> (1990)
%%   \citetext{priv. comm.} ==>> (priv. comm.)
%%   \citenum{key}          ==>> 11 [non-superscripted]
%% Note: full author lists depends on whether the bib style supports them;
%%       if not, the abbreviated list is printed even when full requested.
%%
%% For names like della Robbia at the start of a sentence, use
%%   \Citet{dRob98}         ==>> Della Robbia (1998)
%%   \Citep{dRob98}         ==>> (Della Robbia, 1998)
%%   \Citeauthor{dRob98}    ==>> Della Robbia


%% References with bibTeX database:

\bibliographystyle{model2-names-astronomy}
\bibliography{acndf}

%% Authors are advised to submit their bibtex database files. They are
%% requested to list a bibtex style file in the manuscript if they do
%% not want to use model2-names.bst.

%% References without bibTeX database:

% \begin{thebibliography}{00}

%% \bibitem must have one of the following forms:
%%   \bibitem[Jones et al.(1990)]{key}...
%%   \bibitem[Jones et al.(1990)Jones, Baker, and Williams]{key}...
%%   \bibitem[Jones et al., 1990]{key}...
%%   \bibitem[\protect\citeauthoryear{Jones, Baker, and Williams}{Jones
%%       et al.}{1990}]{key}...
%%   \bibitem[\protect\citeauthoryear{Jones et al.}{1990}]{key}...
%%   \bibitem[\protect\astroncite{Jones et al.}{1990}]{key}...
%%   \bibitem[\protect\citename{Jones et al., }1990]{key}...
%%   \harvarditem[Jones et al.]{Jones, Baker, and Williams}{1990}{key}...
%%

% \bibitem[ ()]{}

% \end{thebibliography}

\end{document}

%%
%% End of file `elsarticle-template-2-harv.tex'.
