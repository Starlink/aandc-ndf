%% This is file `elsarticle-template-2-harv.tex',
%%
%% Copyright 2009 Elsevier Ltd
%%
%% This file is part of the 'Elsarticle Bundle'.
%% ---------------------------------------------
%%
%% It may be distributed under the conditions of the LaTeX Project Public
%% License, either version 1.2 of this license or (at your option) any
%% later version.  The latest version of this license is in
%%    http://www.latex-project.org/lppl.txt
%% and version 1.2 or later is part of all distributions of LaTeX
%% version 1999/12/01 or later.
%%
%% The list of all files belonging to the 'Elsarticle Bundle' is
%% given in the file `manifest.txt'.
%%
%% Template article for Elsevier's document class `elsarticle'
%% with harvard style bibliographic references
%%
%% $Id: elsarticle-template-2-harv.tex 155 2009-10-08 05:35:05Z rishi $
%% $URL: http://lenova.river-valley.com/svn/elsbst/trunk/elsarticle-template-2-harv.tex $
%%

%%\documentclass[preprint,authoryear,12pt]{elsarticle}

%% Use the option review to obtain double line spacing
%% \documentclass[authoryear,preprint,review,12pt]{elsarticle}

%% Use the options 1p,twocolumn; 3p; 3p,twocolumn; 5p; or 5p,twocolumn
%% for a journal layout:

%% Astronomy & Computing uses 5p
%% \documentclass[final,authoryear,5p,times]{elsarticle}
\documentclass[final,authoryear,5p,times,twocolumn]{elsarticle}

%% if you use PostScript figures in your article
%% use the graphics package for simple commands
%% \usepackage{graphics}
%% or use the graphicx package for more complicated commands
\usepackage{graphicx}
%% or use the epsfig package if you prefer to use the old commands
%% \usepackage{epsfig}

%% The amssymb package provides various useful mathematical symbols
\usepackage{amssymb}
%% The amsthm package provides extended theorem environments
%% \usepackage{amsthm}

\usepackage[pdftex,pdfpagemode={UseOutlines},bookmarks,bookmarksopen,colorlinks,linkcolor={blue},citecolor={green},urlcolor={red}]{hyperref}
\usepackage{hypernat}

%% The lineno packages adds line numbers. Start line numbering with
%% \begin{linenumbers}, end it with \end{linenumbers}. Or switch it on
%% for the whole article with \linenumbers after \end{frontmatter}.
%% \usepackage{lineno}

%% natbib.sty is loaded by default. However, natbib options can be
%% provided with \biboptions{...} command. Following options are
%% valid:

%%   round  -  round parentheses are used (default)
%%   square -  square brackets are used   [option]
%%   curly  -  curly braces are used      {option}
%%   angle  -  angle brackets are used    <option>
%%   semicolon  -  multiple citations separated by semi-colon (default)
%%   colon  - same as semicolon, an earlier confusion
%%   comma  -  separated by comma
%%   authoryear - selects author-year citations (default)
%%   numbers-  selects numerical citations
%%   super  -  numerical citations as superscripts
%%   sort   -  sorts multiple citations according to order in ref. list
%%   sort&compress   -  like sort, but also compresses numerical citations
%%   compress - compresses without sorting
%%   longnamesfirst  -  makes first citation full author list
%%
%% \biboptions{longnamesfirst,comma}

% \biboptions{}

\journal{Astronomy \& Computing}

%% Make single quotes look right in verbatim mode
\usepackage{upquote}

\begin{document}

\begin{frontmatter}

%% Title, authors and addresses

%% use the tnoteref command within \title for footnotes;
%% use the tnotetext command for the associated footnote;
%% use the fnref command within \author or \address for footnotes;
%% use the fntext command for the associated footnote;
%% use the corref command within \author for corresponding author footnotes;
%% use the cortext command for the associated footnote;
%% use the ead command for the email address,
%% and the form \ead[url] for the home page:
%%
%% \title{Title\tnoteref{label1}}
%% \tnotetext[label1]{}
%% \author{Name\corref{cor1}\fnref{label2}}
%% \ead{email address}
%% \ead[url]{home page}
%% \fntext[label2]{}
%% \cortext[cor1]{}
%% \address{Address\fnref{label3}}
%% \fntext[label3]{}

\title{The Future of Astronomical Data Formats II. Learning from 25
  years of the $N$-Dimensional Data Format}

%% use optional labels to link authors explicitly to addresses:
%% \author[label1,label2]{<author name>}
%% \address[label1]{<address>}
%% \address[label2]{<address>}

\author[cornell]{Tim Jenness\corref{cor1}}
\ead{tjenness@cornell.edu}
\author[jac]{David S. Berry}
\author[jac]{Malcolm J.\ Currie}
\author[noao]{Frossie Economou}
\author[aao]{Keith Shortridge}

\cortext[cor1]{Corresponding author}

\address[cornell]{Department of Astronomy, Cornell University, Ithaca,
  NY 14853, USA}
\address[jac]{Joint Astronomy Centre, 660 N.\ A`oh\=ok\=u Place, Hilo, HI
  96720, USA}
\address[noao]{National Optical Astronomy
  Observatory, 950 N Cherry Ave, Tucson, AZ 85719, USA}
\address[aao]{Australian Astronomical Observatory, 105 Delhi Rd, North
Ryde, NSW 2113, Australia}


\begin{abstract}
%% Text of abstract

The extensible $N$-dimensional Data Format (NDF) was designed and
developed in the late 1980s to provide a data model suitable for use
in a variety of astronomy data processing applications supported by
Starlink. This paper provides an overview of the historical drivers
for the development of NDF and the lessons learned from using the
format for many years in the Starlink software collection and in data
acquisition systems.

\end{abstract}

\begin{keyword}
%% keywords here, in the form: keyword \sep keyword

%% MSC codes here, in the form: \MSC code \sep code
%% or \MSC[2008] code \sep code (2000 is the default)

data formats \sep
Starlink

\end{keyword}

\end{frontmatter}

% \linenumbers

%% Journal abbreviations
\newcommand{\mnras}{Mon Not R Astron Soc}
\newcommand{\aap}{Astron Astrophys}
\newcommand{\aaps}{Astron Astrophys Supp}
\newcommand{\pasp}{Pub Astron Soc Pacific}
\newcommand{\apj}{Astrophys J}
\newcommand{\apjs}{Astrophys J Supp}
\newcommand{\qjras}{Quart J R Astron Soc}
\newcommand{\an}{Astron.\ Nach.}
\newcommand{\ijimw}{Int.\ J.\ Infrared \& Millimeter Waves}
\newcommand{\procspie}{Proc.\ SPIE}
\newcommand{\aspconf}{ASP Conf. Ser.}

%% Applications


%% main text
\section{Introduction}
\label{sec:intro}

Author list not finalised. Should invite PTW, PWD, NXG and RFWS and others.

Set the scene of the 1980s. FITS \citep{1979ipia.coll..445W,1981A&AS...44..363W} tapes are
the transport medium of choice.

Distinguish data transport from data processing format.

FITS initially adopted by Starlink but data processing demands led to HDS.

Bulk Data Frame (BDF) format used by the Starlink INTERIM environment
\citep[see e.g.][]{1980SPIE..264...70P}. The BDF format was heavily
influenced by FITS and had a special ``IMAGE'' variant standardised
for astronomical images and spectra.


\section{Hierarchical Data System}

First mention I can find of hierarchical file format is in Enterprise
\#4 from June 1981. \citep{1981STARENT4}

Then Disney \& Wallace 1982 \citep{1982QJRAS..23..485D} mention the
system more explicitly.

Lawden 1986 \citep{1986BICDS..30...13L}
 says HDS is great but the Starlink Software Environment is
pain and suffering.

Basic overview of HDS. Presumably no need to talk about internals.

Initial language was BLISS. C investigated but issues with compiler on
VAX when doing system calls prevented its use.
Ported to Unix early 1990s.

Starlink chronology says initial HDS design was August 1981. V1 ready
for testing in June 1982. Version 2 in August 1983. 1981 Confirmed by
Lawden 1991 Bulletin 8, Page 2.

Note in passing that SSN/27 was the HDS reference manual but these
days SSN/27 is staradmin. There is a version of SSN/27 in the obsolete
Starlink document directory from 1989. An interesting document.

Initially called Starlink Data System (changed in October 1983 to
HDS). Presumably \texttt{.sdf} comes from Starlink Data Format?

1991 RFWS ports HDS to Sun. Was this version 3 of the format?


Now supports 64-bit LARGEFILE file offsets (but not 64-bit data arrays
yet -- code is mostly there but has a bug). This was HDS v4. Recently
added 64-bit integer data types.

Automatically detects byte order and does byte swapping in the library.

First couple of pages of SUN/92.

\begin{table}
\caption{HDS basic data types}
\label{tab:hdstypes}
\begin{center}
\begin{tabular}{lll}
\hline
\_BYTE & b & Signed 8-bit integer \\
\_UBYTE & ub & Unsigned 8-bit integer \\
\_WORD & w & Signed 16-bit integer \\
\_UWORD & uw & Unsigned 16-bit integer \\
\_INTEGER & i & Signed 32-bit integer \\
\_INT64 & k &Signed 64-bit integer \\
\_REAL & r & 32-bit float \\
\_DOUBLE & d & 64-bit float \\
\hline
\end{tabular}
\end{center}
\end{table}


\section{Early Years}

The Starlink INTERIM software environment was delivered with BDF format.

IRAS CRDD data (SUN/91) was in .BDF format. Used STARIN to convert
from BDF to HDS file suitable for kappa (SUN/96). STARIN phased out in
1990. Replaced with CONVERT (SUN/55).

Unconstrained use of structures leads to anarchy.

Wright \& Giddings 1983 influence seemed to end up with the
IMAGE data format as described in Appendix H of SUN/95.

\begin{figure*}
\begin{minipage}{\textwidth}
\begin{quote}
\small
\begin{verbatim}
HORSEHEAD  <IMAGE>

   DATA_ARRAY(384,512)  <_REAL>   100.5,102.6,110.1,109.8,105.3,107.6,
                                  ... 123.1,117.3,119,120.5,127.3,108.4
   TITLE          <_CHAR*72>      'KAPPA - Flip'
   DATA_MIN       <_REAL>         28.513
   DATA_MAX       <_REAL>         255.94
\end{verbatim}
\end{quote}
\caption{Example IMAGE structure developed by Wright \& Giddings 1983.}
\end{minipage}
\end{figure*}


Data STructures (DST) format. \citep{1993ASPC...52..219S} see Fig.\
\ref{fig:dst}

The \textsc{Asterix} X-Ray data reduction package \citep{SUN98} used the HDS
format exclusively until the introduction of an abstract data access
interface \citep{1995ASPC...77..199A} which allowed for the use of HDS
and FITS format files. \textsc{Asterix} defined its
own data models and layered those on top of HDS. The \texttt{SPECTRA}
layout was closest to that found in NDF and an example is shown in
Fig.\ \ref{fig:asterix} with a trace from a file created in 1992. The
use of \texttt{HISTORY}, \texttt{AXIS} and \texttt{DATA\_ARRAY}
structures provides some similarity to NDF.

\begin{figure*}
\begin{minipage}{\textwidth}
\begin{quote}
\small
\begin{verbatim}
OUT  <FIGARO>
   Z              <IMAGE>         {structure}
      DATA(310,19)   <_REAL>         1655.552,1376.111,1385.559,1746.966,
                                     ... 1513.654,1465.343,1446.902,*,*,*,*,*
      UNITS          <_CHAR*32>      'A/D numbers per exposure'
      LABEL          <_CHAR*32>      'OBJECT - DARK'
      ERRORS(310,19)  <_REAL>        9.330093,4.624712,1.043125,3.801913,
                                     ... 16.92331,11.49692,10.9114,*,*,*,*,*

   OBS            <OBS>           {structure}
      OBJECT         <_CHAR*32>      'krypton singlet'

   X              <AXIS>          {structure}
      DATA(310)      <_REAL>         1.85115,1.852412,1.853675,1.854938,
                                     ... 2.237606,2.238869,2.240132,2.241395
      LABEL          <_CHAR*32>      'Estimated wavelength'
      UNITS          <_CHAR*32>      'microns'

   Y              <AXIS>          {structure}
      DATA(19)       <_REAL>         0.5,1.5,2.5,3.5,4.5,5.5,6.5,7.5,8.5,
                                     ... 12.5,13.5,14.5,15.5,16.5,17.5,18.5

   FITS           <FITS>          {structure}
      INSTRUME       <_CHAR*8>       'CGS4'
      TELESCOP       <_CHAR*8>       'UKIRT'
      SOFTWARE       <_CHAR*8>       'CF v1.0'
\end{verbatim}
\end{quote}
\caption{Partial dump of the structure of a DST file. This example is from a CGS4
  observation from January 1991.}
\label{fig:dst}
\end{minipage}
\end{figure*}

\begin{figure*}
\begin{minipage}{\textwidth}
\begin{quote}
\small
\begin{verbatim}
RZ_PSPC_Z00000  <SPECTRA>

   AXIS(3)        <AXIS>          {array of structures}

   Contents of AXIS(1)
      DATA_ARRAY(729)  <_REAL>       0.07115,0.07154,0.07162001,0.07166,
                                     ... 2.97,2.975,2.98,2.985,2.99,2.995,3

   EMBOL_UNITS    <_CHAR*16>      'keV/cm**3/s'
   HISTORY        <HISTORY>       {structure}
      CREATED        <_CHAR*18>      '16-JUN-92 01:07:08'
      RECORDS(10)    <HIST_REC>      {array of structures}

      Contents of RECORDS(1)
         DATE           <_CHAR*30>      '   16-JUN-92 01:07:08'
         COMMAND        <_CHAR*35>      'RS_CREATE V1.3        [1 Jun 92]'

      EXTEND_SIZE    <_INTEGER>      10
      CURRENT_RECORD  <_INTEGER>     2

   DATA_ARRAY(729,2,162)  <_REAL>   10183284,2089223,1044320,1044320,
                                    ... 18350.33,18349.45,18275.11,36474.31
   TMIN           <_REAL>         0.017558
   TMAX           <_REAL>         29.13903
   TLSTEP         <_REAL>         0.02
   ENERGY_BOUNDS(730)  <_REAL>    0.07115,0.07154,0.07162001,0.07166,
                                  ... 2.97,2.975,2.98,2.985,2.99,2.995,3,3.01
   REDSHIFT       <_REAL>         0
   EMBOL(2,162)   <_REAL>         1.2466619E9,3.2856078E10,1.1444687E9,
                                  ... 2.8344144E8,2.3983852E9,2.8333168E8
\end{verbatim}
\end{quote}
\caption{Partial dump of the structure of an ASTERIX HDS
  \texttt{spectra} file. This example comes from the ASTERIX source
  distribution from 1992.}
\label{fig:asterix}
\end{minipage}
\end{figure*}

Were there any other structures used with HDS other than DST and IMAGE prior to NDF?

\section{\emph{N}-Dimensional Data Format}

There is text in SGP/38 that can probably be brought over verbatim.

Funnily enough this seems to be a data model and not a data format.

Interestingly \citet{1993ASPC...52..199B} talks of how useful NDF was in designing the
2dF data reduction system using object-oriented fortran with NDF as
the object backing store.

Not trying to be comprehensive covering all possible astronomy
data. Trying to be generically useful whilst providing an extension facility.

Earliest reference to NDF I have found is Starlink bulletin 1988,
number 2. \citep{1988STARB...2...11C}

Anatomy of an NDF file

Proper block diagram of the hierarchy for each structure type? That
was part of the ADASS poster that could not be squeezed into the ADASS
paper and it needs to be made explicit.

\subsection{Data Arrays}

Data, Variance, Quality

Quality labels.

\subsubsection{Data compression}

Short description of options \citep{2008ASPC..394..650C}

\subsection{Character Attributes}

Title, Label and units

\subsection{Axes and WCS}

First there was \texttt{AXIS} and then there was AST. \citep{1998ASPC..145...41W,2001ASPC..238..129B}

\subsection{History}

Structured with a supporting API.

\subsection{Extensions}

\texttt{MORE}

FITS headers as single character block.

\subsubsection{Provenance}

Technically an extension rather than part of NDF. Or talk in context
of history?

Describe how we store the provenance and how this evolved from ascii
format for performance reasons. We also have scaling difficulties.

\section{Data Acquisition}

UKIRT and JCMT. AAO?

\section{Lessons Learned}

Covered somewhat in ADASS paper so need some more content as this is
the justification for the existence of this paper.

Successes:

\begin{itemize}
\item demonstrated usability in many wavelength regimes
\item can round-trip to and from FITS \citep[see e.g.][]{1997STARB..19...14C}
\item Pragmatic adoption of FITS header structure did make the format
  more transparent with the downside being that libraries had to be
  used to parse the character array and keep up with changes to the
  FITS convention.
\item Addition of data compression, WCS objects and provenance has
  proven the flexibility of the early design.
\end{itemize}

Things to be improved upon:

\begin{itemize}
\item Quality array should support more than a byte mask
\item Tables are important
\item Flexible variance definitions. Covariance: See Meyerdierks 1991,
  Starlink Bulletin 8, 19
\item Be vigilant with too much complexity in extensions when reusing
  an NDF structure in an extension can simplify tool access.
\item Checksumming. The FITS DATASUM facility is very useful and NDF
  should support it. Ideally it should be possible to generate a
  reference checksum for a structure. It may be that this has to be
  done in conjunction with HDS.
\end{itemize}


\section{Conclusions}

From its beginnings in the mid-1980s the NDF data model has been used
throughout the Starlink software collection within diverse
applications such as SMURF \citep{2013MNRAS.430.2545C}, CCDPACK, GAIA
and KAPPA. It has also been used as a raw data acquisition format at the
James Clerk Maxwell Telescope and the United Kingdom Infrared
Telescope (was it used at ING and AAO?) and available in that form
from the UKIRT and JCMT archives at the Canadian Astronomy Data Centre
\citep{2008ASPC..394..450E,P01_adassxxiii}. The shift from arbitrary
use of hierarchical structures to a data model enforced by a library
and API is extremely important.

Mention HDX \citep{2003ASPC..295..221G}.

Consider reimplementing NDF on top of HDF5. \texttt{MORE} extension
requires that initially this is done by reimplementing HDS itself as a
layer on top of HDF5. Then all the Starlink software would still work
and even tools like \textsc{hdstrace}. Would allow for an easy way to
add table support to NDF.

The main impediment to further NDF take up is that it's in
Fortran. Rewriting NDF and ARY in C would take some effort.

\section{Acknowledgments}

This research has made use of NASA's Astrophysics Data System.
The Starlink software is currently maintained by the Joint Astronomy
Centre, Hawaii with support from the UK Science and Technology
Facilities Council.

The source code for the NDF library and the Starlink software
(ascl:1110.012) is open-source and is available on github at
\htmladdnormallink{https://github.com/Starlink}.

%% The Appendices part is started with the command \appendix;
%% appendix sections are then done as normal sections
%% \appendix

%% \section{}
%% \label{}

%% References
%%
%% Following citation commands can be used in the body text:
%%
%%  \citet{key}  ==>>  Jones et al. (1990)
%%  \citep{key}  ==>>  (Jones et al., 1990)
%%
%% Multiple citations as normal:
%% \citep{key1,key2}         ==>> (Jones et al., 1990; Smith, 1989)
%%                            or  (Jones et al., 1990, 1991)
%%                            or  (Jones et al., 1990a,b)
%% \cite{key} is the equivalent of \citet{key} in author-year mode
%%
%% Full author lists may be forced with \citet* or \citep*, e.g.
%%   \citep*{key}            ==>> (Jones, Baker, and Williams, 1990)
%%
%% Optional notes as:
%%   \citep[chap. 2]{key}    ==>> (Jones et al., 1990, chap. 2)
%%   \citep[e.g.,][]{key}    ==>> (e.g., Jones et al., 1990)
%%   \citep[see][pg. 34]{key}==>> (see Jones et al., 1990, pg. 34)
%%  (Note: in standard LaTeX, only one note is allowed, after the ref.
%%   Here, one note is like the standard, two make pre- and post-notes.)
%%
%%   \citealt{key}          ==>> Jones et al. 1990
%%   \citealt*{key}         ==>> Jones, Baker, and Williams 1990
%%   \citealp{key}          ==>> Jones et al., 1990
%%   \citealp*{key}         ==>> Jones, Baker, and Williams, 1990
%%
%% Additional citation possibilities
%%   \citeauthor{key}       ==>> Jones et al.
%%   \citeauthor*{key}      ==>> Jones, Baker, and Williams
%%   \citeyear{key}         ==>> 1990
%%   \citeyearpar{key}      ==>> (1990)
%%   \citetext{priv. comm.} ==>> (priv. comm.)
%%   \citenum{key}          ==>> 11 [non-superscripted]
%% Note: full author lists depends on whether the bib style supports them;
%%       if not, the abbreviated list is printed even when full requested.
%%
%% For names like della Robbia at the start of a sentence, use
%%   \Citet{dRob98}         ==>> Della Robbia (1998)
%%   \Citep{dRob98}         ==>> (Della Robbia, 1998)
%%   \Citeauthor{dRob98}    ==>> Della Robbia


%% References with bibTeX database:

\bibliographystyle{model2-names-astronomy}
\bibliography{acndf}

%% Authors are advised to submit their bibtex database files. They are
%% requested to list a bibtex style file in the manuscript if they do
%% not want to use model2-names.bst.

%% References without bibTeX database:

% \begin{thebibliography}{00}

%% \bibitem must have one of the following forms:
%%   \bibitem[Jones et al.(1990)]{key}...
%%   \bibitem[Jones et al.(1990)Jones, Baker, and Williams]{key}...
%%   \bibitem[Jones et al., 1990]{key}...
%%   \bibitem[\protect\citeauthoryear{Jones, Baker, and Williams}{Jones
%%       et al.}{1990}]{key}...
%%   \bibitem[\protect\citeauthoryear{Jones et al.}{1990}]{key}...
%%   \bibitem[\protect\astroncite{Jones et al.}{1990}]{key}...
%%   \bibitem[\protect\citename{Jones et al., }1990]{key}...
%%   \harvarditem[Jones et al.]{Jones, Baker, and Williams}{1990}{key}...
%%

% \bibitem[ ()]{}

% \end{thebibliography}

\end{document}

%%
%% End of file `elsarticle-template-2-harv.tex'.
